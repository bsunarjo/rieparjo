\documentclass[a4paper, DIV11, abstracton]{scrartcl}
\usepackage{color,graphicx}			% graphics package

\usepackage{microtype}				% optical margin alignment
\usepackage[T1]{fontenc}			% umlauts for european languages
\usepackage{textcomp}				% provides extra symbols
\usepackage[utf8]{inputenc}			% character encoding
\usepackage[USenglish]{babel}		% elements language and hyphenation

\usepackage[round]{natbib}			% scientific bibliography
%\usepackage{url}					% nicely spaced urls
\usepackage{amsmath}				% mathematical equation alignment
%\usepackage{dcolumn}				% decimal alignment in tables
\usepackage{siunitx}				% units (micrometers, anyone?)
\usepackage[font=small, format=plain, labelsep=period, labelfont={sf,bf}, justification=justified]{caption}	% formatting the labels for figures, tables

% fonts
\usepackage{palatino}
\usepackage{mathpazo}
\usepackage[scaled=0.92]{helvet}

\setcounter{secnumdepth}{1}			% only number top headings (\section)
\setcounter{tocdepth}{2}			% include \subsection in TOC

\usepackage[raiselinks, pdfborder={0 0 0}]{hyperref}  % creates pdf sections for acrobat
%\hypersetup{pdftitle={Title}}




\begin{document}


\title{Matlab 2011}
\author{Christoph Rieper and Benjamin Sunarjo}


\maketitle
%\thispagestyle{empty} % no page numbers on title page

Document Version: 1.0\\*
Group Name: Rieparjo\\*
Group Members: Christoph Rieper and Benjamin Sunarjo


\setcounter{page}{1}	% start page numbering here
%=-=-=-=-=-=-=-=-=-=-=-=-=-=-=-=-=-=-=-=-=-=-=-=-=-=-=-=-=-=-=-=-=-=
\section{Introduction}
%=-=-=-=-=-=-=-=-=-=-=-=-=-=-=-=-=-=-=-=-=-=-=-=-=-=-=-=-=-=-=-=-=-=

(States your motivation clearly: why is it important / interesting to solve this problem?)
(Add real-world examples, if any)
(Put the problem into a historical context, from what does it originate? Are there already some proposed solutions?)
\citet{helbing.1997}


%=-=-=-=-=-=-=-=-=-=-=-=-=-=-=-=-=-=-=-=-=-=-=-=-=-=-=-=-=-=-=-=-=-=
\section{Fundamental Question}
%=-=-=-=-=-=-=-=-=-=-=-=-=-=-=-=-=-=-=-=-=-=-=-=-=-=-=-=-=-=-=-=-=-=

(At the end of the project you want to find the answer to these questions)
(Formulate a few, clear questions. Articulate them in sub-questions, from the more general to the more specific. )
(Define dependent and independent variables you want to study. Say how you want to measure them.)


%=-=-=-=-=-=-=-=-=-=-=-=-=-=-=-=-=-=-=-=-=-=-=-=-=-=-=-=-=-=-=-=-=-=
\section{Research Methods}
%=-=-=-=-=-=-=-=-=-=-=-=-=-=-=-=-=-=-=-=-=-=-=-=-=-=-=-=-=-=-=-=-=-=

(Cellular Automata, Agent-Based Model, Continuous Modeling…)
(If you are not sure here: 1. Consult your colleagues, 2. ask the teachers, 3. remember that you can change it afterwards)


%=-=-=-=-=-=-=-=-=-=-=-=-=-=-=-=-=-=-=-=-=-=-=-=-=-=-=-=-=-=-=-=-=-=
\section{Expected Results}
%=-=-=-=-=-=-=-=-=-=-=-=-=-=-=-=-=-=-=-=-=-=-=-=-=-=-=-=-=-=-=-=-=-=

(What are the answers to the above questions that you expect to find before starting your research?)



%=-=-=-=-=-=-=-=-=-=-=-=-=-=-=-=-=-=-=-=-=-=-=-=-=-=-=-=-=-=-=-=-=-=
%\addcontentsline{toc}{section}{References}  % adds references section to TOC
\bibliographystyle{apalike2}
\bibliography{references}


\end{document}
